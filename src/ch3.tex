\chapter{The Definition of Groups}
\label{ch:the-definition-of-groups}
\vskip -4em \citep[chapter 3]{pinter_2016}

\begin{enumerate}[label={\Alph*.},font={\bfseries}]
  \item {\bf Examples of Abelian Groups}
    \begin{enumerate}[label={\arabic*},font={\bfseries}]
      \item $\Group{\mathbb{R}, x*y=x+y+k}$
        \begin{enumerate}[label={(\roman*)}]
          \item $*$ is commutative:
            $x*y=x+y+k = y+x+k=y*x$
          \item $*$ is associative.
            \begin{alignat*}{3}
              & x(yz) &&= x(y+z+k) &&= x+y+z+2k \\
              & (xy)z &&= (x+y+k)z &&= (xy)z \\
              & x(yz) &&= (xy)z
            \end{alignat*}
          \item $\mathbb{R}$ has an identity element with respect to $*$.
            \begin{align*}
              xe &= x \\
              x+e+k &= x \\
              e &= -k \\
              (-k)x &= x \\
              -k+x+k &= x
            \end{align*}
          \item $\forall x\in\mathbb{R}(\exists x^\prime\in\mathbb{R}(x*x^\prime=-k))$
            \begin{alignat*}{3}
              xx^\prime &= -k \\
              x+x^\prime+k &= -k \\
              x^\prime &= -x-2k \\
              x^{\prime}x &= xx^\prime & \text{due to commutativity}
            \end{alignat*}
        \end{enumerate}
      \item $\Group{\mathbb{R}^*, x*y=\frac{xy}{2}}$
        \begin{enumerate}[label={(\roman*)}]
          \item $*$ is commutative:
            $x*y=\frac{xy}{2} = \frac{yx}{2}=y*x$
          \item $*$ is associative.
            \begin{alignat*}{3}
              x*(y*z) &= x*(\frac{yz}{2}) &= \frac{xyz}{4} \\
              (x*y)*z &= (\frac{xy}{2})*z &= \frac{xyz}{4}
            \end{alignat*}
          \item $\mathbb{R}^*$ has an identity element with respect to $*$.
            \begin{align*}
              x*e &= \frac{xe}{2} = \frac{ex}{2}=e*x = x \\
              e &= 2
            \end{align*}
          \item $\forall x\in\mathbb{R}(\exists x^\prime\in\mathbb{R}(x*x^\prime=2))$
            \begin{align*}
              x*x^\prime &= \frac{xx^\prime}{2} = \frac{x^{\prime}x}{2} = x^{\prime}*x = e = 2 \\
              x^\prime = \frac{4}{x}
            \end{align*}
        \end{enumerate}
      \item $\Group{\Set{x\in\mathbb{R} : x \neq -1}, x*y=x+y+xy}$
        \begin{enumerate}[label={(\roman*)}]
          \item $*$ is commutative: $x*y=x+y+xy = y+x+yx=y*x$
          \item $*$ is associative.
            \begin{alignat*}{3}
              x*(y*z) &= x*(y+z+yz) = x+(y+z+yz)+x(y+z+yz) &= x+y+z+xy+xz+yz+xyz \\
              (x*y)*z &= (x+y+xy)*z = (x+y+xy)+z+(x+y+xy)z &= x+y+z+xy+xz+yz+xyz
            \end{alignat*}
          \item $\Set{x\in\mathbb{R} : x \neq -1}$ has an identity element with respect to $*$.
            \begin{align*}
              x*e &= x+e+xe = e+x+ex = e*x = x \\
              e(x+1) &= 0 \\
              e &= 0
            \end{align*}
          \item Every element of $\Set{x\in\mathbb{R} : x \neq -1}$ has an inverse with respect to $*$.
            \begin{align*}
              x*x^\prime &= x+x^\prime+xx^\prime=x^\prime+x+x^{\prime}x = e = 0 \\
              x^{\prime}(x+1) &= -x \\
              x^\prime &= -\frac{x}{x+1}
            \end{align*}
        \end{enumerate}
      \item $\Group{\Set{x\in\mathbb{R} :-1 < x < 1}, x*y=\frac{x+y}{xy+1}}$
        \begin{enumerate}[label={(\roman*)}]
          \item $*$ is commutative: $x*y=\frac{x+y}{xy+1}=\frac{y+x}{yx+1}=y*x$
          \item $*$ is associative.
            \begin{alignat*}{4}
              x*(y*z) &= x*(\frac{y+z}{yz+1})
              &= \frac{x+(\frac{y+z}{yz+1})}{x(\frac{y+z}{yz+1})+1}
              &= \frac{xyz+x+y+z}{xy+xz+yz+1} \\
              (x*y)*z &= \frac{x+y}{xy+1}*z
              &= \frac{(\frac{x+y}{xy+1})+z}{(\frac{x+y}{xy+1})z+1}
              &= \frac{x+y+z+xyz}{xy+yz+xz+1}
            \end{alignat*}
          \item $\Set{x\in\mathbb{R} : -1 < x < 1}$ has an identity element w.r.t. $*$.
            \begin{align*}
              x*e &= \frac{x+e}{xe+1} = x \\
              x+e &= x(xe+1) \\
              e &= ex^2 \\
              e(1-x^2) &= 0 \\
              e &= 0 \\
              x*0 &= \frac{x+0}{(x\times0)+1} = x = \frac{0+x}{0x+1} = 0*x
            \end{align*}
          \item Every element of $\Set{x\in\mathbb{R} : -1 < x < 1}$ has an inverse with respect to $*$.
            \begin{align*}
              x * x^\prime &= \frac{x+x^\prime}{xx^\prime+1} = 0; \quad
              x+x^\prime = 0; \quad
              x^\prime = -x \\
              x*(-x) &= \frac{x-x}{x(-x)+1} = 0 = \frac{-x+x}{-x^2+1} = (-x)*x
            \end{align*}
        \end{enumerate}
    \end{enumerate}
  \item {\bf Groups on the Set $\mathbb{R} \times \mathbb{R}$}
    \begin{enumerate}[label={\arabic*},font={\bfseries}]
    \item $(a,b) * (c,d) = (ad + bc, bd)$, on the set $\Set{(x,y)\in\mathbb{R}\times\mathbb{R} : y \ne 0}$
      \begin{enumerate}[label={(\roman*)}]
      \item $*$ is commutative.
        \begin{align*}
          (c,d) * (a,b) &= (cb+da,db) \\
          &= (ad+bc,bd) \\
          &= (a,b) * (c,d)
        \end{align*}
      \item $*$ is associative.
        \begin{align*}
          (a,b) * \left[(c,d) * (e,f)\right] &= (a,b) * (cf+de,df) \\
          &= (adf+bcf+bde,bdf) \\
          &= (ad+bc, bd) * (e,f) \\
          &= \left[(a,b) * (c,d)\right] * (e,f)
        \end{align*}
      \item $(e_1,e_2) = (0,1)$
        \begin{align*}
          (a,b) * (e_1,e_2) &= (ae_2 + be_1, be_2) \\
          &= (a,b) \\
          \\
          be_2 &= b \\
          e_2 &= 1 \\
          \\
          ae_2 + be_1 &= a \\
          a + be_1 &= a \\
          e_1 &= 0
        \end{align*}
      \item $(a^\prime,b^\prime) = \left(\frac{-a}{b^2}, \frac{1}{b}\right)$
        \begin{align*}
          (a,b) * (a^\prime,b^\prime) &= (ab^\prime + ba^\prime, bb^\prime) \\
          &= (0,1)
        \end{align*}
        \begin{align*}
          bb^\prime &= 1 \\
          b^\prime &= \frac{1}{b} \\
          \\
          ab^\prime + ba^\prime &= 0 \\
          \frac{a}{b} + ba^\prime &= 0 \\
          \\
          ba^\prime &= \frac{-a}{b} \\
          a^\prime &= \frac{-a}{b^2} \\
          \\
          (a,b) * \left(\frac{-a}{b^2}, \frac{1}{b}\right) &= \left(\frac{a}{b} + \frac{-a}{b}, b\left(\frac{1}{b}\right)\right) \\
          &= (0,1)
        \end{align*}
      \end{enumerate}
    \end{enumerate}
\end{enumerate}
