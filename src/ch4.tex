\setcounter{chapter}{3}
\chapter{Elementary Properties of Groups}
\label{ch:elementary-properties-of-groups}
%% FIXME: \cite{Pinter2016}*{Chapter 4}

\begin{enumerate}[label={\Alph*.},font={\bfseries}]
\item {\bf Solving Equations in Groups}
  \begin{enumerate}[label={\arabic*},font={\bfseries}]
  \item
    \begin{align*}
      axb &= c \\
      ax &= cb^{-1} \\
      x &= a^{-1}cb^{-1}
    \end{align*}
  \item
    \begin{align*}
      x^2b &= xa^{-1}c \\
      xb &= a^{-1}c \\
      x &= a^{-1}cb^{-1}
    \end{align*}
  \item
    \begin{align*}
      acx &= xac \\
      xacx &= x^2ac
      \\
      x^2a &= bxc^{-1} \\
      x^2ac &= bx \\
      \\
      xacx &= bx \\
      xac &= b \\
      x &= b(ac)^{-1}
    \end{align*}
  \item
    \begin{align*}
      x^3 &= e \\
      \\
      ax^2 &= b \\
      a &= bx \\
      x &= b^{-1}a
    \end{align*}
  \item
    \begin{align*}
      x^5 &= e \\
      x^4 &= x^{-1} \\
      \\
      x^2 &= a^2 \\
      x^4 &= a^2x^2 \\
      x^{-1} &= a^2x^2 \\
      e &= a^4x \\
      \left(a^4\right)^{-1} &= x
    \end{align*}
  \item
    \begin{align*}
      x^2a &= (xa)^{-1} \\
      \\
      (xax)^3 &= bx \\
      xa(x^2a)(x^2a)x &= bx \\
      xa(xa)^{-1}(xa)^{-1}x &= bx \\
      (xa)^{-1}x &= bx \\
      a^{-1}x^{-1}x &= bx \\
      b^{-1}a^{-1} &= x
    \end{align*}
  \end{enumerate}
\item {\bf Rules of Algebra in Groups} \\
  $$G = \Group{\Set{I,A,B,C,D,K}, \cdot}$$ \\
  \begin{alignat*}{4}
    & \mathbf{I} &&=
    \begin{pmatrix}
      \phantom{-}1 & \phantom{-}0 \\
      \phantom{-}0 & \phantom{-}1
    \end{pmatrix}
    \qquad
    \mathbf{A} &&=
    \begin{pmatrix}
      \phantom{-}0 & \phantom{-}1 \\
      \phantom{-}1 & \phantom{-}0
    \end{pmatrix}
    \qquad
    \mathbf{B} &&=
    \begin{pmatrix}
      \phantom{-}0 & \phantom{-}1 \\
      -1 & -1
    \end{pmatrix} \\
    & \mathbf{C} &&=
    \begin{pmatrix}
      -1 & -1 \\
      \phantom{-}0 & \phantom{-}1
    \end{pmatrix}
    \qquad
    \mathbf{D} &&=
    \begin{pmatrix}
      -1 & -1 \\
      \phantom{-}1 & \phantom{-}0
    \end{pmatrix}
    \qquad
    \mathbf{K} &&=
    \begin{pmatrix}
      \phantom{-}1 & \phantom{-}0 \\
      -1 & -1
    \end{pmatrix}
  \end{alignat*}
  \begin{enumerate}[label={\arabic*},font={\bfseries}]
  \item
    \(
      \mathbf{A}^2 =
      \begin{pmatrix}
        0 & 1 \\
        1 & 0
      \end{pmatrix}
      \begin{pmatrix}
        0 & 1 \\
        1 & 0
      \end{pmatrix}
      =
      \begin{pmatrix}
        1 & 0 \\
        0 & 1
      \end{pmatrix}
      = e
    \)
    ... but $\mathbf{A} \ne e$, so $x^2 = e \centernot\implies x = e$.\qed
  \item $\mathbf{A}^2 = \mathbf{I}^2$, but $\mathbf{A} \ne \mathbf{I}$, so $x^2 = a^2 \centernot\implies x = a$.\qed
  \item $(\mathbf{A}\mathbf{B})^2 = \mathbf{K}^2 = \mathbf{I}$, and $\mathbf{A}^2\mathbf{B}^2 = \mathbf{I}\mathbf{D} = \mathbf{D}$, but $\mathbf{I} \ne \mathbf{D}$, so $(ab)^2 = a^2b^2$ is not true in every group $G$.\footnote{$(ab)^2 = a^2b^2$ is only true in abelian groups.}\qed
  \item $x^2 = x \implies x = e$
    \begin{align*}
      x^2 &= x \\
      xx &= x \\
      xxx^{-1} &= xx^{-1} \\
      xe &= e \\
      x &= e
    \end{align*}
    \qed
  \item
    \captionof{table}{$\Group{\Set{I,A,B,C,D,K}, \cdot}$}
    \begin{tabular}{ c | c c c c c c}
      $\cdot$ & $\mathbf{I}$ & $\mathbf{A}$ & $\mathbf{B}$ & $\mathbf{C}$ & $\mathbf{D}$ & $\mathbf{K}$ \\
      \hline
      $\mathbf{I}$ & $\mathbf{I}$ \\
      $\mathbf{A}$ & & $\mathbf{I}$ \\
      $\mathbf{B}$ & & & $\mathbf{D}$ \\
      $\mathbf{C}$ & & & & $\mathbf{I}$ \\
      $\mathbf{D}$ & & & & & $\mathbf{B}$ \\
      $\mathbf{K}$ & & & & & & $\mathbf{I}$
    \end{tabular}

    As shown in the table, there does not exist an $x \in G$ such that $x = y^2$ for $y \in \Set{\mathbf{A}, \mathbf{C}, \mathbf{K}}$. \\
    Therefore $\neg\left(\forall x \in G,\ \exists y \in G\left( x = y^2 \right)\right)$.\qed
  \item
    \begin{align*}
      y &= xz \\
      x^{-1}y &= x^{-1}xz \\
      z &= x^{-1}y
    \end{align*}
    Therefore, for all $x,y \in G$, there exists a $z \in G$ such that $y = xz$.\qed
  \end{enumerate}
\newpage
\item {\bf Elements That Commute}
  \begin{enumerate}[label={\arabic*},font={\bfseries}]
  \item $a^{-1}b^{-1} = (ba)^{-1} = (ab)^{-1} = b^{-1}a^{-1}$
  \item Since $a = b^{-1}ba = b^{-1}ab$, $ab^{-1} = (b^{-1}ab)b^{-1} = b^{-1}a$.
  \item $a(ab) = a(ba) = (ab)a$
  \item $a^2b^2 = b^2a^2$
    \begin{proof}
      First, Assume $a^2b^2 = b^2a^2$, that is $aabb = bbaa$.
      Since $a$ and $b$ commute, $abab = baba$, $abba = abba$, $baba = abab$, and $bbaa = aabb$.
      %%
      Next, assume $bbaa = aabb$.
      Since $a$ and $b$ commute, $baba = abab$, $abba = baab$, $abab = baba$, and $aabb = bbaa$.
    \end{proof}
  \item $(xax^{-1})(xbx^{-1}) = xa(x^{-1}x)bx^{-1} = x(ab)x^{-1} = x(ba)x^{-1} = xb(x^{-1}x)ax^{-1} = (xbx^{-1})(xax^{-1})$
  \item $ab = ba \iff ab{a\inverse} = b$
    \begin{proof}
      First, assume $ab = ba$.
      Multiplying by $a^{-1}$ on the right shows $ab = ba \implies aba^{-1} = b$.
      %%
      Next, assume $aba^{-1} = b$.
      Multiplying by $a$ on the right shows $aba^{-1} = b \implies ab = ba$.
    \end{proof}
  \item $ab = ba \iff ab{a\inverse}{b\inverse} = e$
    \begin{proof}
      First, assume $ab = ba$.
      Multiplying by $a^{-1}$ on the right yields $aba^{-1} = b$.
      Then multiplying by $b^{-1}$ on the right yields $aba^{-1}b^{-1} = e$.
      Thus $ab = ba \implies x$.
      %%
      Next, assume $aba^{-1}b^{-1} = e$.
      Multiplying by $b$ on the right yields $aba^{-1} = b$.
      Then multiplying by $a^{-1}$ on the right yields $ab = ba$.
      Thus $aba^{-1}b^{-1} = e \implies ab = ba$ and $ab = ba \iff aba^{-1}b^{-1} = e$.
    \end{proof}
  \end{enumerate}
\item {\bf Group Elements and Their Inverses}
  \begin{enumerate}[label={\arabic*},font={\bfseries}]
  \item $ab = e \implies ba = e$
    \begin{proof}
      If $ab = e$, then $ab = aa^{-1}$, so by the cancellation law, $b = a^{-1}$ and $a = b^{-1}$. \\
      Thus, $bb^{-1} = e \implies ba = e$, as desired.
    \end{proof}
  \item $abc = e \implies cab = e$ and $bca = e$.
    \begin{proof}
      If $(ab)c = e$, then $(ab)c = (ab)(ab)^{-1}$, so by the cancellation law, $c = (ab)^{-1} = b^{-1}a^{-1}$ \\
      Thus, $(ab)^{-1}(ab) = e \implies c(ab) = e$, and $b(b^{-1}a^{-1})a = e \implies cba = e$.
    \end{proof}
  \item ...
  \item Let $G$ be a group such that $xay = a^{-1}$ for all $a,x,y \in G$.
    Prove that $yax = a^{-1}$ as well.
    \begin{proof}
      If $xay = a^{-1}$, then $(xay)a = a^{-1}a$, so by the definition of inversion, $(xay)a = e$.
      Thus $x^{-1}(xay)ax = x^{-1}ex$, so by associativity and the definition of the identity element, $(x^{-1}x)a(yax) = e \iff ea(yax) = e \iff a(yax) = e$.
      Multiply by $a^{-1}$ on the left to obtain $a^{-1}a(yax) = a^{-1}e$, so by the definition of inversion, $yax = a^{-1}$.
    \end{proof}
  \item Let $a = a^{-1}$, $b = b^{-1}$, and $c = c^{-1}$. If $ab = c$ show that $bc = a$ and $ca = b$ as well.
    \begin{align*}
      ab &= c \\
      abb^{-1} &= cb^{-1} = cb \\
      a &= cb \\
      a^{-1} &= b^{-1}c^{-1} = bc \\
      bc &= a \\
      \\
      ab &= c \\
      b^{-1}a^{-1} &= c^{-1} \\
      ba^{-1} &= c \\
      ba^{-1}a &= ca \\
      ca &= b
    \end{align*}
  \item Let $abc = (abc)^{-1}$, show that $bca = (bca)^{-1}$ and $cab = (cab)^{-1}$.
    \begin{align*}
      abc &= (abc)^{-1} \\
      bca &= a^{-1}(abc)^{-1}a \\
      &= a^{-1}(bc)^{-1} \\
      &= (bca)^{-1} \\
      \\
      bca &= (bca)^{-1} \\
      cab &= b^{-1}(bca)^{-1}b \\
      &= b^{-1}(ca)^{-1} \\
      &= (cab)^{-1}
    \end{align*}
  \item Let $a = a^{-1}$ and $b = b^{-1}$, show that $(ab)^{-1} = ba$.
    \begin{proof}
      Replace $a$ and $b$ with their inverses on the right-hand side of
      $(ab)^{-1} = b^{-1}a^{-1}$ to obtain $(ab)^{-1} = ba$.
    \end{proof}
  \item $a = a^{-1} \iff a^2 = e$
    \begin{proof}
      If $a = a^{-1}$, then $a^2 = e$ by multiplying by $a$ on the right.
      If $a^2 = e$, then $a = a^{-1}$ by multiplying by $a^{-1}$ on the right.
    \end{proof}
  \item Let $c = c^{-1}$. Prove $ab = c \iff abc = e$.
    \begin{proof}
      If $ab = c$, then $ab = c^{-1}$, since $c = c^{-1}$.
      Multiply by $c$ on the right to obtain $abc = e$.
      %%
      If $abc = e$, then $abc^{-1} = e$ since $c = c^{-1}$.
      Multiply by $c$ on the right to obtain $ab = c$.
    \end{proof}
  \end{enumerate}
\item {\bf Counting Elements and Their Inverses}
  \begin{enumerate}[label={\arabic*},font={\bfseries}]
  \item Prove that in any finite group $G$, $2 \mid \left|\Set{x \in G : x \ne x\inverse}\right|$.
    \begin{proof}
      By definition, $G = \Set{x \in G : x = x\inverse} \cup \Set{x \in G : x \ne x\inverse}$.

      Therefore, $\forall x \in G\left(x = x\inverse \lor \left(x \ne x\inverse \land \exists y \in G\left(y \ne x \land y = x\inverse\right)\right)\right)$.

      So, $\left|\Set{x \in G : x \ne x\inverse}\right| = \left|\Set{x_0,x_0\inverse,x_1,x_1\inverse,x_2,x_2\inverse,x_3,x_3\inverse...}\right| = 2k$.
    \end{proof}
  \item Prove $\left|\Set{x \in G : x = x\inverse}\right|$ has the same parity as $|G|$.
    \begin{proof}
      Since $|G| = \left|\Set{x \in G : x = x\inverse}\right| + \left|\Set{x \in G : x \ne x\inverse}\right|$, \\
      and $\left|\Set{x \in G : x \ne x\inverse}\right|$ is even, $\left|\Set{x \in G : x = x\inverse}\right|$ has the same parity as $|G|$.
    \end{proof}
  \item Prove $2 \mid |G| \implies \exists x \in G\left(x \ne e \land x = x\inverse \right)$.
    \begin{proof}
      If $2 \mid |G|$ then $2 \mid \left|\Set{x \in G : x = x\inverse}\right|$. Since $e = e\inverse$, $2 \nmid \left|\Set{x \in G : x \ne e \land x = x\inverse}\right|$ and thus $\exists x \in G\left(x \ne e \land x = x\inverse\right)$.
    \end{proof}
  \item Given a finite abelian group $G = \Set{e,a_1,a_2,...a_n}$, prove $(a_1a_2...a_n)^2 = e$.
    \begin{align*}
      (a_1a_2...a_n)^2 &= (a_1a_2...a_n)(a_1^{-1}a_2^{-1}...a_n^{-1}) \\
      &= a_1a_1^{-1}a_2a_2^{-1}...a_na_n^{-1} \\
      &= ee...e \\
      &= e
    \end{align*}
    \qed
  \item Prove $\forall x \in G\left(x \ne e \implies x \ne x\inv\right) \implies a_1a_2...a_n = e$.
    \begin{proof}
      Assume $\forall x \in G\left(x \ne e \implies x \ne x\inv\right)$.
      Then $\forall x \in a_1a_2...a_n\left(\exists y \in a_1a_2...a_n\left(x \ne y \land y = x\inv\right)\right)$.
      So $a_1a_2...a_n$ can be rewritten $a_1a_1^{-1}a_2a_2^{-}...a_{n/2}a_{n/2}^{-1}$, which reduces to $e$.
    \end{proof}
  \item Prove that if there is exactly one $x \ne e$ in $G$ such that $x = x\inv$ then $a_1a_2...a_n = x$.
    \begin{proof}
      $a_1a_2...a_n$ can be rewritten $xa_1a_1^{-1}a_2a_2^{-1}...a_{n/2}a_{n/2}^{-1}$, which is equivalent to $xe$.
    \end{proof}
  \end{enumerate}
\item {\bf Constructing Small Groups}
  \begin{enumerate}[label={\arabic*},font={\bfseries}]
  \item $a,b \in G$
    \begin{enumerate}[label={(\alph*)},font={\bfseries}]
    \item Prove $a^2 = a \implies a = e$.
      \begin{proof}
        Assume $a^2 = a$. Divide by $a$ to get $a = e$.
      \end{proof}
    \item Prove $ab = a \implies b = e$.
      \begin{proof}
        Assume $ab = a$.
        Multiply by $a\inv$ on the left to get ${a\inv}ab = {a\inv}a \equiv b = e$.
      \end{proof}
    \item Prove $ab = b \implies a = e$.
      \begin{proof}
        Assume $ab = b$.
        Multiply by $b\inv$ on the right to get $ab{b\inv} = b\inv \equiv a = e$.
      \end{proof}
    \end{enumerate}
  \item ...\todo[inline]{Explain why elements of each row in a Cayley table must be distinct.}
  \item There is exactly one group with three distinct elements.
    \captionof{table}{\href{https://nathancarter.github.io/group-explorer/Multtable.html?groupURL=https://nathancarter.github.io/group-explorer/groups/Z_3.group}{Multiplication Table for $\mathbb{Z}_3$}}
    \begin{tabular}{ c | c c c }
      $\cdot$ & $\mathbf{e}$ & $\mathbf{a}$ & $\mathbf{b}$ \\
      \hline
      $\mathbf{e}$ & $e$ & $a$ & $b$ \\
      $\mathbf{a}$ & $a$ & $b$ & $e$ \\
      $\mathbf{b}$ & $b$ & $e$ & $a$
    \end{tabular}
    %% \newpage
  \item There is exactly one group $G$ with four elements, such that $\forall x \in G (xx = e)$.
    \captionof{table}{\href{https://nathancarter.github.io/group-explorer/Multtable.html?groupURL=https://nathancarter.github.io/group-explorer/groups/V_4.group}{Multiplication Table for $V_4$}}
    \begin{tabular}{ c | c c c c }
      $\cdot$ & $\mathbf{e}$ & $\mathbf{a}$ & $\mathbf{b}$ & $\mathbf{c}$  \\
      \hline
      $\mathbf{e}$ & $e$ & $a$ & $b$ & $c$ \\
      $\mathbf{a}$ & $a$ & $e$ & $c$ & $b$ \\
      $\mathbf{b}$ & $b$ & $c$ & $e$ & $a$ \\
      $\mathbf{c}$ & $c$ & $b$ & $a$ & $e$
    \end{tabular}
  \item There is exactly one group $G$ with four elements, such that $\exists x \in G (x \ne e \land xx = e)$ and
    $\exists y \in G (yy \ne e)$. $bb = e$ and $aa = cc = e$.
    \captionof{table}{\href{https://nathancarter.github.io/group-explorer/Multtable.html?groupURL=https://nathancarter.github.io/group-explorer/groups/Z_4.group}{Multiplication Table for $\mathbb{Z}_4$}}
    \begin{tabular}{ c | c c c c }
      $\cdot$ & $\mathbf{e}$ & $\mathbf{a}$ & $\mathbf{b}$ & $\mathbf{c}$  \\
      \hline
      $\mathbf{e}$ & $e$ & $a$ & $b$ & $c$ \\
      $\mathbf{a}$ & $a$ & $b$ & $c$ & $e$ \\
      $\mathbf{b}$ & $b$ & $c$ & $e$ & $a$ \\
      $\mathbf{c}$ & $c$ & $e$ & $a$ & $b$
    \end{tabular}
    \item \todo[inline]{Explain why $V_4$ and $\mathbb{Z}_4$ are the only possible groups of order 4.}
  \end{enumerate}
\item {\bf Direct Products of Groups}
  \begin{enumerate}[label={\arabic*},font={\bfseries}]
  \item Prove that $G \times H$ is a group.
    \begin{proof}
      \ \\
      \begin{enumerate}[label={(G\arabic*)}]
      \item
        \begin{align*}
          (x_1,y_1)\left[(x_2,y_2)(x_3,y_3)\right] &= (x_1,y_1)(x_2x_3,y_2y_3) \\
          &= (x_1x_2x_3,y_1y_2y_3) \\
          &= (x_1x_2,y_1y_2)(x_3,y_3) \\
          &= \left[(x_1,y_1)(x_2,y_2)\right](x_3,y_3)
        \end{align*}
      \item Let $e_G$ be the identity element of $G$, and $e_H$ the identity element of $H$.
        The identity element of $G \times H$ is $(e_G,e_H)$.
        \begin{alignat*}{3}
          &(x,y)(e_G,e_H) &&= (xe_G,ye_H) &&= (x,y) \\
          \\
          &(e_G,e_H)(x,y) &&= (e_Gx,e_Hy) &&= (x,y)
        \end{alignat*}
      \item $\forall (a,b) \in G \times H \left( (a,b)\inv = (a\inv,b\inv) \right)$
        \begin{alignat*}{4}
          &(a,b)(a\inv,b\inv) &&= (aa\inv,bb\inv) &&= (e_G,e_H) &&= e_{G \times H}\\
          \\
          &(a\inv,b\inv)(a,b) &&= ({a\inv}a,{b\inv}b) &&= (e_G,e_H) &&= e_{G \times H}
        \end{alignat*}
      \end{enumerate}
    \end{proof}
  \item $\mathbb{Z}_2 \times \mathbb{Z}_3 = \Set{(0,0),(0,1),(0,2),(1,0),(1,1),(1,2)}$
    \captionof{table}{$\mathbb{Z}_2 + \mathbb{Z}_3$}
    \begin{tabular}{ c | c c c }
      $+$ & $0$ & $1$ & $2$ \\
      \hline
      $0$ & $(0,0)$ & $(0,1)$ & $(0,2)$ \\
      $1$ & $(1,0)$ & $(1,1)$ & $(1,2)$ \\
      $2$ & $(2,0)$ & $(2,1)$ & $(2,2)$
    \end{tabular}
  \item If $G$ and $H$ are abelian, prove that $G \times H$ is abelian.
    \begin{proof}
      Assume $G$ is abelian, i.e. $\forall a,c \in G (a+c = c+a)$. Similarly, assume $H$ is abelian, i.e. \\
      $\forall b,d \in H (b+d = d+b)$.
      Therefore $\forall a,c \in G\ b,d \in G (a+c,b+d) = (c+a,b+d)$, i.e. $(a,b)+(c,d) = (c,d)+(a,b)$, so $G \times H$ is abelian.
      %% Let $\pi_G : G \times G \to G$ and $\pi_H : G \times H \to H$ be the projection \glspl{homomorphism}. \\
      %% \begin{tikzcd}[sep=large]
      %%   P\dar[swap]{f_H}\drar[dashed]{f}\rar{f_G} & G \\
      %%   H & {G \times H}\lar{\pi_H}\uar[swap]{\pi_G}
      %% \end{tikzcd}
    \end{proof}
  \item Suppose $\forall g \in G\ g = g\inv$ and $\forall h \in H\ h = h\inv$. Prove that $\forall (g,h) \in G \times H\ (g,h) = (g,h)\inv$.
    \begin{proof}
      \begin{align*}
        (g,h)(g,h) &= (g,h)(g\inv,h\inv) \\
        &= (gg\inv,hh\inv) \\
        &= (e_G,e_H) = e_{G \times H} \\
        &= (g,h)(g,h)\inv \\
        (g,h) &= (g,h)\inv
      \end{align*}
    \end{proof}
  \end{enumerate}
  %% \newpage
\item {\bf Powers and Roots of Group Elements}
  \begin{enumerate}[label={\arabic*},font={\bfseries}]
  \item Prove by induction that $\forall n \in \mathbb{N}\ (bab\inv)^n = ba^nb\inv$.
    \begin{proof}
      When $n=1$, $(bab\inv)^1 = ba^1b\inv$. Next, assume $(bab\inv)^n = ba^nb\inv$.
      \begin{align*}
        (bab\inv)^{n+1} &= (bab\inv)^n(bab\inv) \\
        &= (ba^nb\inv)(bab\inv) \\
        &= ba^n({b\inv}b)ab\inv \\
        &= ba^nab\inv \\
        &= ba^{n+1}b\inv
      \end{align*}
    \end{proof}
  \item Prove by induction that $\forall n \in \mathbb{N}\ ab = ba \implies (ab)^n = a^nb^n$.
    \begin{proof}
      When $n=1$, $(ab)^1 = ab = a^1b^1$. Next, assume $(ab)^n = a^nb^n$.
      \begin{align*}
        (ab)^{n+1} &= (ab)^n(ab) \\
        &= (a^nb^n)(ab) \\
        &= a^nb^nab \\
        &= a^nab^nb \\
        &= a^{n+1}b^{n+1}
      \end{align*}
    \end{proof}
  \item $xax = e \implies (xa)^{2n} = a^n$
    \begin{proof}
      When $n=1$, $(xa)^2 = xaxa = ea = a = a^1$. Next, assume $(xa)^{2n} = a^n$.
      \begin{align*}
        (xa)^{2n+2} &= (xa)^{2n}(xa)^2 \\
        &= a^n(xa)^2 \\
        &= a^n(xax)a \\
        &= a^na \\
        &= a^{n+1}
      \end{align*}
    \end{proof}
  \item If $a^3 = e$, then $a$ has a square root, i.e. $a = x^2$.
    \begin{proof}
      Assume $a^3 = e$. Divide by $a^2$ to get $a = (a\inv)^2$.
    \end{proof}
  \item If $a^2 = e$, then $a$ has a cube root, i.e. $a = x^3$.
    \begin{proof}
      Assume $a^2 = e$. Divide by $a^{-1}$ on both sides to get $a = a\inv$.
      Then multiply by $a^2$ on the right to get $aa^2 = {a\inv}a^2$,  or equivalently, $a = a^3$.
    \end{proof}
  \item If $a\inv = x^3$ then $a = y^3$
    \begin{proof}
      Assume $a\inv = x^3$, i.e. $a\inv = xxx$. Then $(a\inv)\inv = (xxx)\inv$, i.e. $a = ({x\inv})^3$.
    \end{proof}
  \item If $x^2ax = a\inv$ then $a = y^3$.
    \begin{proof}
      Assume $x^2ax = a\inv$. Rewrite $(xax)^3$ as $xa(x^2ax)xax$.
      Replace $x^2ax$ with $a\inv$ to get $xa{a\inv}xax = x^2ax$.
      Therefore $(xax)^3 = a\inv$, and so $a = \left((xax)^3\right)\inv$.
    \end{proof}
  \item If $xax = b$, then $ab = y^2$.
    \begin{proof}
      Assume $xax = b$. Then $ab = a(xax) = (ax)^2$.
    \end{proof}
  \end{enumerate}
\end{enumerate}
