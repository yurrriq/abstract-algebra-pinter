\chapter{Elementary Properties of Groups}
\label{ch:elementary-properties-of-groups}
%% FIXME: \cite{Pinter2016}*{Chapter 4}

\begin{enumerate}[label={\Alph*.},font={\bfseries}]
\item {\bf Solving Equations in Groups}
  \begin{enumerate}[label={\arabic*},font={\bfseries}]
  \item
    \begin{align*}
      axb &= c \\
      ax &= cb^{-1} \\
      x &= a^{-1}cb^{-1}
    \end{align*}
  \item
    \begin{align*}
      x^2b &= xa^{-1}c \\
      xb &= a^{-1}c \\
      x &= a^{-1}cb^{-1}
    \end{align*}
  \item
    \begin{align*}
      acx &= xac \\
      xacx &= x^2ac
      \\
      x^2a &= bxc^{-1} \\
      x^2ac &= bx \\
      \\
      xacx &= bx \\
      xac &= b \\
      x &= b(ac)^{-1}
    \end{align*}
  \item
    \begin{align*}
      x^3 &= e \\
      \\
      ax^2 &= b \\
      a &= bx \\
      x &= b^{-1}a
    \end{align*}
  \item
    \begin{align*}
      x^5 &= e \\
      x^4 &= x^{-1} \\
      \\
      x^2 &= a^2 \\
      x^4 &= a^2x^2 \\
      x^{-1} &= a^2x^2 \\
      e &= a^4x \\
      \left(a^4\right)^{-1} &= x
    \end{align*}
  \item
    \begin{align*}
      x^2a &= (xa)^{-1} \\
      \\
      (xax)^3 &= bx \\
      xa(x^2a)(x^2a)x &= bx \\
      xa(xa)^{-1}(xa)^{-1}x &= bx \\
      (xa)^{-1}x &= bx \\
      a^{-1}x^{-1}x &= bx \\
      b^{-1}a^{-1} &= x
    \end{align*}
  \end{enumerate}
\item {\bf Rules of Algebra in Groups} \\
  $$G = \Group{\Set{I,A,B,C,D,K}, \cdot}$$ \\
  \begin{alignat*}{4}
    & \mathbf{I} &&=
    \begin{pmatrix}
      \phantom{-}1 & \phantom{-}0 \\
      \phantom{-}0 & \phantom{-}1
    \end{pmatrix}
    \qquad
    \mathbf{A} &&=
    \begin{pmatrix}
      \phantom{-}0 & \phantom{-}1 \\
      \phantom{-}1 & \phantom{-}0
    \end{pmatrix}
    \qquad
    \mathbf{B} &&=
    \begin{pmatrix}
      \phantom{-}0 & \phantom{-}1 \\
      -1 & -1
    \end{pmatrix} \\
    & \mathbf{C} &&=
    \begin{pmatrix}
      -1 & -1 \\
      \phantom{-}0 & \phantom{-}1
    \end{pmatrix}
    \qquad
    \mathbf{D} &&=
    \begin{pmatrix}
      -1 & -1 \\
      \phantom{-}1 & \phantom{-}0
    \end{pmatrix}
    \qquad
    \mathbf{K} &&=
    \begin{pmatrix}
      \phantom{-}1 & \phantom{-}0 \\
      -1 & -1
    \end{pmatrix}
  \end{alignat*}
  \begin{enumerate}[label={\arabic*},font={\bfseries}]
  \item
    \(
      \mathbf{A}^2 =
      \begin{pmatrix}
        0 & 1 \\
        1 & 0
      \end{pmatrix}
      \begin{pmatrix}
        0 & 1 \\
        1 & 0
      \end{pmatrix}
      =
      \begin{pmatrix}
        1 & 0 \\
        0 & 1
      \end{pmatrix}
      = e
    \)
    ... but $\mathbf{A} \ne e$, so $x^2 = e \centernot\implies x = e$.\qed
  \item $\mathbf{A}^2 = \mathbf{I}^2$, but $\mathbf{A} \ne \mathbf{I}$, so $x^2 = a^2 \centernot\implies x = a$.\qed
  \item $(\mathbf{A}\mathbf{B})^2 = \mathbf{K}^2 = \mathbf{I}$, and $\mathbf{A}^2\mathbf{B}^2 = \mathbf{I}\mathbf{D} = \mathbf{D}$, but $\mathbf{I} \ne \mathbf{D}$, so $(ab)^2 = a^2b^2$ is not true in every group $G$.\footnote{$(ab)^2 = a^2b^2$ is only true in abelian groups.}\qed
  \item $x^2 = x \implies x = e$
    \begin{align*}
      x^2 &= x \\
      xx &= x \\
      xxx^{-1} &= xx^{-1} \\
      xe &= e \\
      x &= e
    \end{align*}
    \qed
  \item
    \captionof{table}{$\Group{\Set{I,A,B,C,D,K}, \cdot}$}
    \begin{tabular}{ c | c c c c c c}
      $\cdot$ & $\mathbf{I}$ & $\mathbf{A}$ & $\mathbf{B}$ & $\mathbf{C}$ & $\mathbf{D}$ & $\mathbf{K}$ \\
      \hline
      $\mathbf{I}$ & $\mathbf{I}$ \\
      $\mathbf{A}$ & & $\mathbf{I}$ \\
      $\mathbf{B}$ & & & $\mathbf{D}$ \\
      $\mathbf{C}$ & & & & $\mathbf{I}$ \\
      $\mathbf{D}$ & & & & & $\mathbf{B}$ \\
      $\mathbf{K}$ & & & & & & $\mathbf{I}$
    \end{tabular}

    As shown in the table, there does not exist an $x \in G$ such that $x = y^2$ for $y \in \Set{\mathbf{A}, \mathbf{C}, \mathbf{K}}$. \\
    Therefore $\neg\left(\forall x \in G,\ \exists y \in G\left( x = y^2 \right)\right)$.\qed
  \item
    \begin{align*}
      y &= xz \\
      x^{-1}y &= x^{-1}xz \\
      z &= x^{-1}y
    \end{align*}
    Therefore, for all $x,y \in G$, there exists a $z \in G$ such that $y = xz$.\qed
  \end{enumerate}
\newpage
\item {\bf Elements That Commute}
  \begin{enumerate}[label={\arabic*},font={\bfseries}]
  \item $a^{-1}b^{-1} = (ba)^{-1} = (ab)^{-1} = b^{-1}a^{-1}$
  \item Since $a = b^{-1}ba = b^{-1}ab$, $ab^{-1} = (b^{-1}ab)b^{-1} = b^{-1}a$.
  \item $a(ab) = a(ba) = (ab)a$
  \item $(xax^{-1})(xbx^{-1}) = xa(x^{-1}x)bx^{-1} = x(ab)x^{-1} = x(ba)x^{-1} = xb(x^{-1}x)ax^{-1} = (xbx^{-1})(xax^{-1})$
  \item $ab = ba \iff aba^{-1} = b$
    \begin{proof}
      First, assume $ab = ba$.
      Multiplying by $a^{-1}$ on the right shows $ab = ba \implies aba^{-1} = b$.
      %%
      Next, assume $aba^{-1} = b$.
      Multiplying by $a$ on the right shows $aba^{-1} = b \implies ab = ba$.
    \end{proof}
  \item $ab = ba \iff aba^{-1}b^{-1} = e$
    \begin{proof}
      First, assume $ab = ba$.
      Multiplying by $a^{-1}$ on the right yields $aba^{-1} = b$.
      Then multiplying by $b^{-1}$ on the right yields $aba^{-1}b^{-1} = e$.
      Thus $ab = ba \implies x$.
      %%
      Next, assume $aba^{-1}b^{-1} = e$.
      Multiplying by $b$ on the right yields $aba^{-1} = b$.
      Then multiplying by $a^{-1}$ on the right yields $ab = ba$.
      Thus $aba^{-1}b^{-1} = e \implies ab = ba$ and $ab = ba \iff aba^{-1}b^{-1} = e$.
    \end{proof}
  \end{enumerate}
\item {\bf Group Elements and Their Inverses}
  \begin{enumerate}[label={\arabic*},font={\bfseries}]
  \item $ab = e \implies ba = e$
    \begin{proof}
      If $ab = e$, then $ab = aa^{-1}$, so by the cancellation law, $b = a^{-1}$ and $a = b^{-1}$. \\
      Thus, $bb^{-1} = e \implies ba = e$, as desired.
    \end{proof}
  \item $abc = e \implies cab = e$ and $bca = e$.
    \begin{proof}
      If $(ab)c = e$, then $(ab)c = (ab)(ab)^{-1}$, so by the cancellation law, $c = (ab)^{-1} = b^{-1}a^{-1}$ \\
      Thus, $(ab)^{-1}(ab) = e \implies c(ab) = e$, and $b(b^{-1}a^{-1})a = e \implies cba = e$.
    \end{proof}
  \item ...
  \item Let $G$ be a group such that $xay = a^{-1}$ for all $a,x,y \in G$.
    Prove that $yax = a^{-1}$ as well.
    \begin{proof}
      If $xay = a^{-1}$, then $(xay)a = a^{-1}a$, so by the definition of inversion, $(xay)a = e$.
      Thus $x^{-1}(xay)ax = x^{-1}ex$, so by associativity and the definition of the identity element, $(x^{-1}x)a(yax) = e \iff ea(yax) = e \iff a(yax) = e$.
      Multiply by $a^{-1}$ on the left to obtain $a^{-1}a(yax) = a^{-1}e$, so by the definition of inversion, $yax = a^{-1}$.
    \end{proof}
  \item Let $a = a^{-1}$, $b = b^{-1}$, and $c = c^{-1}$. If $ab = c$ show that $bc = a$ and $ca = b$ as well.
    \begin{align*}
      ab &= c \\
      abb^{-1} &= cb^{-1} = cb \\
      a &= cb \\
      a^{-1} &= b^{-1}c^{-1} = bc \\
      bc &= a \\
      \\
      ab &= c \\
      b^{-1}a^{-1} &= c^{-1} \\
      ba^{-1} &= c \\
      ba^{-1}a &= ca \\
      ca &= b
    \end{align*}
  \item Let $abc = (abc)^{-1}$, show that $bca = (bca)^{-1}$ and $cab = (cab)^{-1}$.
    \begin{align*}
      abc &= (abc)^{-1} \\
      bca &= a^{-1}(abc)^{-1}a \\
      &= a^{-1}(bc)^{-1} \\
      &= (bca)^{-1} \\
      \\
      bca &= (bca)^{-1} \\
      cab &= b^{-1}(bca)^{-1}b \\
      &= b^{-1}(ca)^{-1} \\
      &= (cab)^{-1}
    \end{align*}
  \item Let $a = a^{-1}$ and $b = b^{-1}$, show that $(ab)^{-1} = ba$.
    \begin{proof}
      Replace $a$ and $b$ with their inverses on the right-hand side of
      $(ab)^{-1} = b^{-1}a^{-1}$ to obtain $(ab)^{-1} = ba$.
    \end{proof}
  \item $a = a^{-1} \iff a^2 = e$
    \begin{proof}
      If $a = a^{-1}$, then $a^2 = e$ by multiplying by $a$ on the right.
      If $a^2 = e$, then $a = a^{-1}$ by multiplying by $a^{-1}$ on the right.
    \end{proof}
  \item Let $c = c^{-1}$. Prove $ab = c \iff abc = e$.
    \begin{proof}
      If $ab = c$, then $ab = c^{-1}$, since $c = c^{-1}$.
      Multiply by $c$ on the right to obtain $abc = e$.
      %%
      If $abc = e$, then $abc^{-1} = e$ since $c = c^{-1}$.
      Multiply by $c$ on the right to obtain $ab = c$.
    \end{proof}
  \end{enumerate}
\end{enumerate}
