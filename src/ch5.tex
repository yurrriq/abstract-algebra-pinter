\setcounter{chapter}{4}
\chapter{Subgroups}
\label{ch:subgroups}
%% FIXME: \cite{Pinter2016}*{Chapter 5}

\begin{enumerate}[label={\Alph*.},font={\bfseries}]
\item {\bf Recognizing Subgroups}
  \begin{enumerate}[label={\arabic*},font={\bfseries}]
  \item $G = \Group{\mathbb{R}, +}, H = \Set{\log a : a \in \mathbb{Q}_{+}}$.
    $H \le G$.
    \begin{enumerate}[label={(\roman*)}]
    \item If $\log a, \log b \in H$, then $\log a + \log b = \log ab$. But $ab \in \mathbb{Q}_{+}$, so $\log ab \in H$.
    \item If $\log a \in H$, then $-(\log a) = \log \frac{1}{a}$. But $\frac{1}{a} \in \mathbb{Q}_{+}$, so $\log \frac{1}{a} \in H$.
    \end{enumerate}
  \item $G = \Group{\mathbb{R}, +}, H = \Set{\log n : n \in \mathbb{Z}_{+}}$.
    $H \not\le G$.
    \begin{enumerate}[label={(\roman*)}]
    \item If $\log n, \log m \in H$, then $\log n + \log m = \log nm$. But $nm \in \mathbb{Z}_{+}$, so $\log nm \in H$.
    \item If $\log n \in H$, then $-(\log n) = \log \frac{1}{n}$. But $\frac{1}{n} \not\in \mathbb{Z}_{+}$, so $\log \frac{1}{n} \not\in H$.
    \end{enumerate}
  \item $G = \Group{\mathbb{R}, +}, H = \Set{x \in \mathbb{R} : \tan x \in \mathbb{Q}}$.
    \todo[inline]{$H \overset{?}{<} G$.}
    \begin{enumerate}[label={(\roman*)}]
    \item If $\tan x, \tan y \in H$, then $\tan x + \tan y$. \todo[inline]{Finish this}
    \end{enumerate}
  \item $G = \Group{\mathbb{R}^{*}, \cdot}, H = \Set{2^n3^m : m,n \in \mathbb{Z}}$.
    \todo[inline]{$H \overset{?}{<} G$.}
    \begin{enumerate}[label={(\roman*)}]
    \item If $2^a3^b, 2^c3^d \in H$, then $2^a3^b2^c3^d = ...$ \todo[inline]{Finish this}
    \end{enumerate}
  \item $G = \Group{\mathbb{R} \times \mathbb{R}, +}, H = \Set{(x,y) : y = 2x}$.
    $H \le G$.
    \begin{enumerate}[label={(\roman*)}]
    \item If $(a,b), (c,d) \in \mathbb{R} \times \mathbb{R}$, then $(a,b) + (c,d) = (a+c,b+d)$.
      But $b+d = 2(a+c)$, so $(a+c,b+d) \in H$.
    \item If $(a+c,b+d) \in H$, then $(-(a+c),-(b+d))$. But $-(b+d) = -2(a+c)$, so $(-(a+c),-(b+d)) \in H$.
    \end{enumerate}
  \item $G = \Group{\mathbb{R} \times \mathbb{R}, +}, H = \Set{(x,y) : x^2 + y^2 > 0}$.
    \todo[inline]{$H \overset{?}{<} G$.}
    \begin{enumerate}[label={(\roman*)}]
    \item If $(a,b), (c,d) \in \mathbb{R} \times \mathbb{R}$, then $(a,b) + (c,d) = (a+c,b+d)$.
      \todo[inline]{But $b+d = ...$, so $...$.}
      \begin{align*}
        (a+c)^2 + (b+d)^2 &\overset{?}{>} 0 \\
        a^2 + c^2 + 2ac + b^2 + d^2 + 2bd &\overset{?}{>} 0 \\
        \\
        a^2 + b^2 &> 0 \\
        c^2 + d^2 &> 0
      \end{align*}
    \end{enumerate}
  \item Prove $C \subseteq D \implies P_C < P_D$.\todo[inline]{Prove it.}
  \end{enumerate}
\item {\bf Subgroups of Groups of Functions}
  \begin{enumerate}[label={\arabic*},font={\bfseries}]
  \item $G = \Group{\mathscr{F}(\mathbb{R}), +}, H = \Set{f \in \mathscr{F}(\mathbb{R}) : f(x) = 0\ \text{for every}\ x \in [0,1]}$
    \begin{enumerate}[label={(\roman*)}]
    \item Suppose $f,g \in H$; then $\forall x \in [0,1]$ $f(x) = 0$ and $g(x) = 0$, so $[f+g](x) = f(x) + g(x) = 0 + 0 = 0$. Thus, $f + g \in H$.
    \item If $f \in H$, then $\forall x \in [0,1]\ f(x) = 0$. Thus $[-f](x) = -f(x) = -0 = 0$, so $-f \in H$.
    \end{enumerate}
  \item $G = \Group{\mathscr{F}(\mathbb{R}), +}, H = \Set{f \in \mathscr{F}(\mathbb{R}) : f(-x) = -f(x)}$
    \begin{enumerate}[label={(\roman*)}]
    \item Suppose $f,g \in H$; then $f(-x) = -f(x)$ and $g(-x) = -g(x)$, so $[f+g](-x) = -f(x) - g(x) = f(-x) + g(-x)$. Thus, $f + g \in H$.
    \item If $f \in H$, then $f(-x) = -f(x)$. Thus $[-f](x) = -f(x) = f(-x)$, and so $-f \in H$.
    \end{enumerate}
  \item $G = \Group{\mathscr{F}(\mathbb{R}), +}, H = \Set{f \in \mathscr{F}(\mathbb{R}) : f \text{is periodic of period} \pi}$
    \begin{enumerate}[label={(\roman*)}]
    \item \todo[inline]{Show that $f + g \in H$}
    \item \todo[inline]{Show that $-f \in H$}
    \end{enumerate}
  \item \todo[inline]{$G = \Group{\mathscr{G}(\mathbb{R}), +}, H = \Set{f \in \mathscr{G}(\mathbb{R}) : \int_0^1 f(x)\ dx = 0}$}
  \item \todo[inline]{$G = \Group{\mathscr{D}(\mathbb{R}), +}, H = \Set{f \in \mathscr{D}(\mathbb{R}) : df/dx\text{ is constant}}$}
  \item \todo[inline]{$G = \Group{\mathscr{F}(\mathbb{R}), +}, H = \Set{f \in \mathscr{F}(\mathbb{R}) : f(x) \in \mathbb{Z}\text{ for every }x \in \mathbb{R}}$}
  \end{enumerate}
\item {\bf Subgroups of Abelian Groups} \\
  In the following exercises, let $G$ be an abelian group.
  \begin{enumerate}[label={\arabic*},font={\bfseries}]
  \item
    \begin{theorem}
      $H = \Set{x \in G : x = x\inv} \implies H < G$
    \end{theorem}
    \begin{proof}
      Assume $y, z \in H$, thus $y = y\inv$ and $z = z\inv$.

      \begin{align*}
        yz &= {y\inv}{z\inv} && \text{(substitute $x = x\inv$)} \\
        &= {z\inv}{y\inv} && \text{(since $G$ is Oberlin)}\\
        &= (yz)\inv && \text{(rewrite ${b\inv}{a\inv} = (ab)\inv$)}
      \end{align*}
      So, $H$ is closed under the group operation of $G$.

      \begin{align*}
        \forall x \in H(x = x\inv) &\implies x \in H \land x\inv \in H
      \end{align*}
      So, $H$ is closed under the inverse operation of $G$.
    \end{proof}
  \item
    \begin{theorem}
      $n \in \mathbb{Z} \land H = \Set{x \in G : x^n = e} \implies H < G$
    \end{theorem}
    \begin{proof}
      Let $n \in \mathbb{Z}$ and $x, y \in H$, thus $x^n = e$ and $y^n = e$.

      \begin{align*}
        x^ny^n &= ee && \text{(substitute $x^n = e$ and $y^n = e$)} \\
        &= e \\
        &= (xy)^n && \text{(since $G$ is abelian)}
      \end{align*}
      So, $H$ is closed under the group operation of $G$.

      \begin{align*}
        (x\inv)^n &= (x^n)\inv \\
        &= e\inv && \text{(substitute $x^n = e$)} \\
        &= e
      \end{align*}
      So, $H$ is closed under the inverse operation of $G$.
    \end{proof}
  \item
    \begin{theorem}
      $H = \Set{x \in G : x = y^2\text{ for some }y \in G} \implies H < G$
    \end{theorem}
    \begin{proof}
      Let $a, b \in H$, so $a = c^2$, $b = d^2$.

      \begin{align*}
        ab &= c^2d^2 \\
        &= (cd)^2 && \text{(since $G$ is abelian)}
      \end{align*}
      So, $H$ is closed under the group operation of $G$.

      \begin{align*}
        a\inv &= (c^2)\inv \\
        &= (c\inv)^2
      \end{align*}
      So, $H$ is closed under the inverse operation of $G$.
    \end{proof}
  \item
    \begin{theorem}
      $H < G \land K = \Set{x \in G : x^2 \in H} \implies K < G$
    \end{theorem}
    \begin{proof}
      Let $a, b \in K$, so $a = c^2$ and $b = d^2$, and $c^2, d^2 \in G$.
      \begin{align*}
        ab &= c^2d^2 \\
        &= (cd)^2 && \text{(since $G$ is abelian)}
      \end{align*}
      So, $K$ is closed under the group operation of $G$.

      \begin{align*}
        a\inv &= (c^2)\inv \\
        &= (c\inv)^2
      \end{align*}
      So, $K$ is closed under the inverse operation of $G$.
    \end{proof}
  \item \todo[inline]{Prove $H < G \land K = \Set{x \in G : \exists n \in \mathbb{Z}^+(x^n \in H)} \implies K < G$.}
  \item \todo[inline]{Prove $H < G \land K < G \land HK = \Set{xy : x \in H \land y \in K} \implies HK < G$.}
  \item \todo[inline]{Explain why parts 4-6 are not true if $G$ is not abelian.}
  \end{enumerate}
\item {\bf Subgroups of an Arbitrary Group}
  \begin{enumerate}[label={\arabic*},font={\bfseries}]
  \item
    \begin{theorem}
      $H < G \land K < G \implies H \cap K < G$
    \end{theorem}

    \begin{align*}
      \forall a,b \in H(ab \in H) && \text{(since $H < G$)} \\
      \forall a,b \in K(ab \in K) && \text{(since $K < G$)} \\
      \forall a, b \in H \cap K(ab \in H \cap K) && \text{(since $ab \in H \land ab \in K$)}
    \end{align*}
    So, $H \cap K$ is closed under the group operation of $G$.

    \begin{align*}
      \forall c \in H \cap K(c \in H \land c \in K \implies c\inv \in H \land c\inv \in K \implies c\inv \in H \cap K) && \text{(since $H, K < G$)}
    \end{align*}
    So, $H \cap K$ is closed under the inverse operation of $G$.
  \item \todo[inline]{...}
  \item \todo[inline]{...}
  \item \todo[inline]{...}
  \item \todo[inline]{...}
  \item \todo[inline]{...}
  \item
    \begin{enumerate}[label={(\alph*)}]
    \item \todo[inline]{...}
    \item \todo[inline]{...}
    \end{enumerate}
  \item
    \begin{enumerate}[label={(\alph*)}]
    \item \todo[inline]{...}
    \item \todo[inline]{...}
    \end{enumerate}
  \end{enumerate}
\end{enumerate}
